\chapter{Tecnologias}\label{cap_tecnologias}

\section{Tecnologias}
Neste trabalho foram usadas diversas tecnologias para auxiliar no processo de desenvolvimento. Esta aplicação também poderia ser projetada para telas web, uma vez que o back-end foi desenvolvido para aceitar requisições web e salvar ou realizar pesquisas. 

\subsection{Java 8 e Mavan}
Dentre estas tecnologias, usamos a linguagem Java na versão 8 (ou superior) conforme utilizada em sala de aula e Maven como ferramenta de automação de compilação.

 "O Maven utiliza um arquivo XML (POM) para descrever o projeto de software sendo construído, suas dependências sobre módulos e componentes externos."\cite{maven}. 

\subsection{Spring e Lombok}
É utilizado do Spring para realizar a conexão com o banco de dados e expor os endpoints utilizados na aplicação, juntamente com o Lombok que é utilizado para facilitar o desenvolvimento evitando ter que digitar todos os getters e setters. Com isso, disponibilizando um builder para facilitar a maneira de instanciar um objeto.

\subsection{MongoDB e ORM}
Foi utilizado o banco de dados não relacional Mongodb com ORM (Object Related Model), "ORM ou Object Relational Mapping é uma técnica de mapeamento de objeto relacional que visa criar uma camada de mapeamento entre nosso modelo de objetos (aplicação) e nosso modelo relacional (banco de dados) de forma a abstrair o acesso ao mesmo."\cite{orm} onde as classes em Java se tornam tabelas no banco de dados e é possível realizar consultas através da classe que extende o banco de dados.

\subsection{IDE Intellij}
Também foi utilizado da IDE IntelliJ para facilitar o desenvolvimento, "O IDE e um programa de computador, geralmente utilizado para aumentar a produtividade dos desenvolvedores de software, bem como a qualidade desses produtos. Podem auxiliar,
através de ferramentas e características, na redução de erros e na aplicação de técnicas
como o RAD (Rapid Application Development)" \cite{ide}.

\subsection{GitHub}
Também foi feito o uso da ferramenta de controle de versão GitHub para gerenciar os arquivos de desenvolvimento e também deste relatório.

