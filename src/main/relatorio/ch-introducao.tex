\chapter{Introdução}\label{cap_intro}

O objetivo deste projeto foi realizar um sistema de academia visando facilitar o processo de controle que diversas academias enfrentam devido aos diversos tipos de pagamentos possiveis neste caso o controle de pagamentos. Para isso, o usuário tem a possibilidade de efetuar cadastro dos alunos, modalidades, valores, data de pagamento entre outras informações. Uma vez que os dados estejam cadastrados, o usuário pode procurar a situação (paga/pendente) dos alunos e realizar o controle e baixa no sistema.

Sendo assim, o Capítulo 2 aborda o processo de desenvolvimento do software. Este, por sua vez, é subdividido: Capítulo 2.1 referente às telas e o Capítulo 2.2 referente às ferramentas usadas e particularidades do back-end.

 Por fim, o Capítulo~\ref{cap_conclu} destaca as considerações finais a respeito do projeto que engloba e utiliza os conceitos vistos durante as aulas de Linguagens Comerciais de Programação lecionadas no primeiro semestre de dois mil e dezenove.
